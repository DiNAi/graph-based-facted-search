\chapter{OnlyFacets exploratory search}
This section deals with the process of exploratory search and its implementation. To begin with, an introductory example is presented and the functionality of the prototype graphical user interface (GUI) is explained.

\section{The user interface for exploratory search}
The graphical user interface is designed to comprise three main areas: the direct search results in the center of column, the {\it faceted filter} on the right, and the exploratory search navigation on the left. The search results include a timeline, which shows the automatically generated temporal segmentation of the results including highlighted segments indicating search hits. The facet filter allows to narrow down the search results based on graph search. 

Faceted search aims to broaden the scope of search by suggesting related terms, concepts and resources. Our approach uses priority queue and data graph to support the search process by exposing additional information about indexed resources.

Figure 1 depicts the result of a query after the user clicking on the facets - restaurant and place. The exploratory search GUI suggests a list of related entities. When the user clicks multiple facets, the labels of the mapped entities grouped by their connecting properties. 

By clicking on, e.g. {\it 'BLACHBL} in the faceted search GUI, a new search is issued and the GUI switches to the newly selected entity showing its related entities and properties (cf. Fig. 2). This supplementary information includes, as e. g., related places (birth place, work place, etc), predecessor and successor in the presidential office, or Barack Obama�s residence.

To retain previous actions, a history list (4) provides links to previous searches. Optionally, the user may activate an additional preview of the search results evoked by a related entity when clicking on it (5). Moving the mouse pointer over these previews causes a popup to show brief information about the video resource (6).

\section{Process workflow of faceted search}

To enable exploratory search as outlined in the previous sections a three-step procedure has been devised: (1) each facet of the user query is mapped to one hits queue, (2) mapped entities are cross-checked with the repository, and (3) for each resulting entity the most imported related resources are determined.

The following sections encompass the workflow stages in detail.

\subsection{Mapping Queries to  Hits Queue}

To map interest score, a hits queue is generated within the interaction processing while the users keep clicking on the different facets. The intermediate data is rather computational intensive, as users can dynamically change their preferences during the time. The hits queue is implemented as a priority queue and comprises a list . Furthermore, the number of hits, when clicking for the entity in the facets search, is stored as \mathrm{freq({md})}. Hence, the gazetteer is defined as set of (metadata, hits):{({md}, freq{md)}.

\subsetion{Converting interest score to weight}
As mentioned in chapter 2, to determine the weight score of the data graph by computing how often the metadata is clicked by the user , we need to map the computed metadata to the weight. To obtain this, 


Related entities are determined by the application of a heuristic based on ranking for users interest score of a given facet f. For a given entity \it e a related resource is initially defined as:
r = {meta, {si}, l, rank}
with meta representing the metadata of the related entity \mathrm{r}, \mathrm{{si}} the synset for the metadata, 