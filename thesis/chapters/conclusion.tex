\chapter{Conclusion and futrue work}
\label{conclusion}

In this work, we have addressed the problem of how to improve faceted search for navigational and exploratory search by using implicit feedback mechanism and demonstrated an improved exploratory search with an evaluation of the search process. We have show how to use graph based mode to enable a simple faceted search. By using this, we were able to make implicitly existing relations among multiple data sets explicit and to augment the ordinary keyword-based search by presenting additional related information and resources to the user via an appropriate interactive user interface.

Faceted search is at it's early stages as a research data. Currently, there does not exist an overall accepted best-practice neither on how to realize nor on how to evaluate. Although, we have obviously increased the recall of obtained results by providing an faceted search interface, the precision of the suggested resources has to be determined by the user and her personal information needs.

Improvements of the graphical user interface explicitly supporting the investigative and navigational aspect of our approach will be considered in future work. For better support in data space navigation, future work is focussed on the combination of faceted and explorative search features to satisfy the searchers curiosity and to foster serendipitous discovery.

Overall, we have implemented a first prototype for exploratory faceted search, which gives the user the possibility to discover resources that are usually hidden aways from the user's eyes in the search engine index.