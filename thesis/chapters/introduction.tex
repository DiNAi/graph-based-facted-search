\chapter{Introduction}
\label{introduction}

Information seekers trying to gauge public opinion or learn about current events face a torrent of information from tens of millions of sources worldwide. Many existing search engines equate user information needs with a keyword query, assuming that a user knows what words to use to best describe his or her information need. However, many common information retrieval tasks do not �t into this traditional keyword search paradigm. Some information needs are not naturally representable as queries. In other cases, an information need may have a natural query, but is too complex to be expressed as keywords. 

Today, keyword-based search engine is arranged manually - suitable keywords or keyword chains are assigned to information resources by trained experts -, sophisticated algorithms are used to generate keywords automatically form (textual) information resources, e.g. researchers have proposed some approaches trying to use user profiles for search engine to provide personalized search result \cite{Tan2006, Shen2005}. However, assigning appropriate keywords remains expert knowledge, i.e. the ordinary user hardly knows anything about the keywords, which are require to actually find a specific resource. Even worse, the user can never be sure about the completeness and the integrity of the achieved search results.

Part of the responsibility for that situation bears the traditional keyword-based search paradigm. You have to know the appropriate keywords to find a specific resource. That's all. But, not all search engine users have the same information needs, because users might have different ways to search for information. Moreover, if the user tries to achieve an overview of actually available information about a certain topic, today's web search engines are flooding search results by millions. Traditional keyword-based search does not consider user's higher level abstraction desire and result ranking is mainly based on link popularity.

Faceted search has gained great success in exploration domain over the past years, and most popular online urban guide websites, such as Yelp, now provides faceted search interfaces. On faceted-search-enabled websites, users can narrow down the list of interesting places by putting constraints on a group of merchandize facets, such as categories, services, reviews, products, etc. Well designed faceted search has been shown to be understood by the average user\cite{Hearst2009}.  However, faceted search is also used as auxiliary search tool for the keyword search. This is also known as cold start problem which involve a degree of automated data modeling. Specifically, it concerns the issue that the system cannot draw any inferences for users or items about which it has not gathered sufficient information. 

This motivates us to explore whether we can adapt the faceted search idea to the general purpose document retrieval. Users might have preferences for certain document facets. For example, online buyers might have preferences on brands, colors, etc. In all these cases, users have clear ideas about some facets of their interested documents, and this information might help the system learn users' preferences and interests. Ideally, users would provide structured queries to describe their information needs more accurately. However, when a user browses the web at different times, he or she could be accessing pages that pertain to different topics. Different categorical data cannot represent a different purpose for a user. However, different kinds of interests might be motivated by the same kind of interest at a higher abstraction level. That is, a user might possess interests at different abstraction levels, and the higher-level interests are more general, while the lower-level ones are more specific.

In this paper, we explore a implicit interactive user feedback mechanism based on facets to solve the cold start problem. In this mechanism, instead of letting users provide relevance feedback on documents or create structured queries actively, the system models the documents as a weighted graph, where the weights are the similarity scores based on users interests. To achieve this goal, the system suggests the faceted constraints (in the form of facet-value pairs) and user can choose interesting facet-value pairs to improve the returned documents.

The proposed faceted feedback mechanism may have the following advantages. First, the suggested facet-value pairs are usually short and easy to understand. Compared with document-based feedback, this may reduce the cognitive overload of the user and thus is more likely to be adopted by the average user. Users can quickly select multiple facet-value pairs in a short time, so the system might get more user feedback. Second, it may help a user better understand the corpus, how the engine works, and train users in how to form better queries.

The rest of this paper is organized as follows. In section 2, we talk about the related work. Section 3 is the focus of this paper, and describes the faceted feedback mechanism. We propose four facet-value pair recommendation methods and two retrieval models in this section. In section 4, we describe the methodology of our experiments. Section 5 gives the experimental results and the corresponding analysis. Section 6 concludes this paper.
