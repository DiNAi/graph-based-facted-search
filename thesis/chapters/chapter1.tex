%\begin{savequote}[75mm]
%Nulla facilisi. In vel sem. Morbi id urna in diam dignissim feugiat. Proin %molestie tortor eu velit. Aliquam erat volutpat. Nullam ultrices, diam tempus %vulputate egestas, eros pede varius leo.
%\qauthor{Quoteauthor Lastname}
%\end{savequote}

\chapter{Related Work}

%Document-based relevance feedback is one of the most widely used explicit feedback mechanisms. In this scenario, users are asked to provide feedback on the relevance of delivered documents. Many approaches have been proposed to incorporate document relevance feedback into retrieval. 

Recommender systems suffer from the cold start problem of a new user who start with an empty profile and encounters a difficulty of communication with his community members. Many approaches have been proposed \cite{Leung2007, Gao2002} as the approach of Z. Zaier\cite{Zaier2010} who has studied the challenges of recommender systems namely the cold start. In our work, we leverage the graph method for exploring related entities based on the user's interest score. However, there are little researches based on graphs for the cold start problem in faceted search area. 

According to the same perspective and to solve the problem of cold start document, Senjuti Basu Roy and Haidong Wang \cite{Roy2008} have proposed another approach based on minimum effort drill-down. Roy proposed to ask searchers to make relevance judgements about returned objects and then executing a revised query based on the judgement. This solution provides a dialog with the user to extract more information from her on other desired attribute values. This approach depends on user interaction which requires strong human participation in a more continuous and exploratory process. However practice shows that people are often unwilling to take the added step to provide feedbacks when the search paradigm is the classic turn-taking model. Our developed approach differs from this prior work along several key dimensions: (a) our proposed approach considers implicit feedback based on users' choice, (b) our proposed approach is graph based and depends on user interaction, (c) our algorithms can work in conjunction with available ranking functions.

Relevance search feedback is a commonly used query refinement technique that can be traced back to 1960s. The basic idea is to reply on user interactions to better capture the user information need. Document-based relevance feedback is one of the most widely used explicit feedback mechanisms. Many approaches have been proposed to incorporate document relevance feedback into retrieval \cite{Zhai2001, Zhang2004}. Our work is motivated by early work in relevance feedback, and differs by focusing on implicitly retrieving user interest score with faceted metadata.

%The greatest advantage of faceted search is that it meet the information seeker's desire for a user interface that organized search results into meaningful groups, in order to help make sense of the results, and to help decide what to do next \cite{Hearst2006}. 

% For an example of a full page figure, see Fig.~\ref{fig:myFullPageFigure}.
